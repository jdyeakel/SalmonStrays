\documentclass[ucm,12pt]{ucletter}


\usepackage{setspace}
\usepackage{graphicx}

\setlength{\parindent}{5ex}
\linespread{1.2}

%\name{Justin D. Yeakel}
\telephone{(209)~285-9571}
\email{jyeakel@ucmerced.edu}

\begin{document}
\begin{letter}{
    \\ Philosophical Transaction of the Royal Society B: Biological Sciences\\
    The Royal Society\\
    6-9 Carlton House Terrace\\
    London England SW1Y 5AG\\ \\
    \vspace{1mm}
    \centerline{\bf{Re: The eco-evolutionary impacts of collective straying on metapopulation robustness}} \\
}


\opening{To the Editorial Board at \emph{Philosophical Transactions},}



%Intro
Please find attached our manuscript entitled ``The eco-evolutionary impacts of collective straying on metapopulation robustness'' co-authored by Justin Yeakel, Jean Philippe Gibert, Peter Westley, and Jonathan Moore, which we would like to submit for publication in the Philosophical Transactions of the Royal Society B themed issue \emph{Collective movement in ecology: From emerging technologies to management implications}, under the section title `Linking collective movement with populations, ecology and evolution'.


A longstanding question in ecology and evolutionary biology concerns how selective gradients and dispersal influences the robustness and persistence of spatially connected metapopulations.
Of particular interest to this themed issue is the role of dispersal among populations that collectively navigate where such dispersal -- or straying for salmonid metapopulations -- may vary as a function of population density (Berdahl et al., \emph{Fish and Fisheries}, 2016).
Because populations interact between habitats that are distributed across a selective mosaic, changes in population size, and by extension the rate of straying, may be strongly influenced by the evolutionary dynamics of trait complexes that determine population-level fitness.

In the accompanying manuscript, we introduce an eco-evolutionary model of metapopulation dynamics that takes into account density dependent straying as well as local selection on a generalized trait complex determining the rate of recruitment.
With this framework, we investigate how increased rates of straying and trait heritability alter dynamic expectations of the metapopulation, and relate this to two measures of metapopulation robustness: the portfolio effect (PE) and the time required for a population to recover following an induced disturbance.
The qualitative results of our model are in accord with empirical observations, and we make a number of predictions that we believe have important implications for conservation.
We make several specific contributions:


\begin{itemize}
  \item Increased straying between two locally-adapted populations results in the emergence of alternative stable states: one \emph{dominant state} population that contains a larger proportion of biomass, and one \emph{subordinate state} population.
  The trait means of both populations are skewed towards that of the dominant population, and this has a large effect on the resultant dynamics.
  \item There is a nonlinear effect of straying rates (either constant or density dependent) on both the portfolio effect and the recovery time: portfolio effects are generally maximized and recovery rates are generally minimized with a low-intermediate rate of straying -- particularly in the alternative steady state regime -- and this promotes robustness.
  % \item We show that the portfolio effect and recovery time following a large disturbance, two independent measures of robustness that concern steady state and transient dynamics, respectively, are strongly correlated.
  \item Increased habitat heterogeneity tends to increase robustness when rates of straying are low, but decrease robustness when rates are straying become large. Importantly, metapopulation robustness is generally increased by density dependent straying, regardless of habitat heterogeneity.
  \item If highly dissimilar habitats are linked by a very low rate of straying (as might be expected if two sites are very distant from one another), we find that the extinction of the dominant population 
\end{itemize}


Our theory is analytic, relatively simple, and synthesizes a variety of fundamental ecological theories (energy equivalence hypothesis, resource competition theory, Cope’s rule, the fasting endurance hypothesis) and shows how many of these are the natural consequence of a simple dynamical framework that incorporates the energetic consequences of starvation and recovery combined with allometric timescales. 
We believe that our findings will be of general interest to ecologists, evolutionary biologists, and paleoecologists, such that consideration for publication in a broad interest journal such as \emph{Philosophical Transactions} is warranted.

We appreciate your consideration of our manuscript.


\vspace{5mm}

\singlespacing
\closing{Sincerely,\\
\fromsig{\includegraphics[scale=0.2]{signature.jpg}}\\
\fromname{
Justin D. Yeakel,\\
Jean Philippe Gibert,\\
Peter A. H. Westley,\\
Jonathan W. Moore}
}

\end{letter}
\end{document}
