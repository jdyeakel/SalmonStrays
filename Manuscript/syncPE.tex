\documentclass{article}


\begin{document}

\subsection{Synchronization and the Portfolio Effect}

In essence, synchronization of subpopulation trajectories over time represents an absence of dynamical diversity within the aggregate.
In contrast, the diversity of the aggregate can be described in terms of the number of subpopulations from which it is composed, and this is more generally referred to as metapopulation diversity.
These two forms of diversity are in fact closely related.
Loreau defined synchronization in terms of the sizes of fluctuations among the components of an aggregate, ${\rm VAR}(N_i)$, to the aggregate itself, ${\rm VAR}(N_T)$, which varies between 0 (not synchronized) and 1 (synchronized), such that

\begin{equation}
\label{eq_sync}
\phi = \frac{{\rm VAR}(N_T)}{\left(\sum_{i=1}^N\sqrt{{\rm VAR}(N_i)}\right)^2}.
\end{equation}

The diversity of the aggregate can defined with respect to the stability that such diversity confers.
The stability conferred by aggregation, which is often called `the portfolio effect' (PE), is defined by a relative decrease in the magnitude of fluctuations for the aggregate, with respect to the magnitude of fluctuations for the components of the aggregate.
Averaged over each subpopulation, the expected portfolio effect can be expressed as

\begin{equation}
\label{eq_PE}
\langle{\rm PE}\rangle =\frac{1}{X}\sum_{i=1}^X \frac{\sqrt{{\rm VAR}(N_i)}}{{\rm E}(N_i)}\cdot \frac{{\rm E}(N_T)}{\sqrt{{\rm VAR}(N_T)}},
\end{equation}

\noindent which is equivalently the average CV of the subpopulations divided by the CV of the aggregate.
Accordingly, as the CV of the aggregate decreases relative to that of the subpopulations, the stability of the aggregate increases, and $\langle{\rm PE}\rangle > 1$.
By rearranging and substituting eq. (\ref{eq_sync}) into eq. (\ref{eq_PE}), we can observe that $\langle{\rm PE}\rangle \propto \sqrt{\phi}^{-1}$.
Thus, the dynamical diversity of an aggregate offers a mirror to the stability conferred by aggregation, where perfect synchrony ($\phi = 1$) reflects the absence of a portfolio effect, or $\langle{\rm PE}\rangle = 1$.
Moreover, if and only if $\phi \neq 1$, it is expected that as the diversity of the aggregate increases, so should the portfolio effect, and this central thesis is thought to underlie the stability of populations, species, communities and even financial markets.


\end{document}