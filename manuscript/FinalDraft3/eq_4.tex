\documentclass[a4paper]{article}

%% Language and font encodings
\usepackage[english]{babel}
\usepackage[utf8x]{inputenc}
\usepackage[T1]{fontenc}

%% Sets page size and margins
\usepackage[a4paper,top=3cm,bottom=2cm,left=3cm,right=3cm,marginparwidth=1.75cm]{geometry}

%% Useful packages
\usepackage{amsmath}
\usepackage{graphicx}
\usepackage[colorinlistoftodos]{todonotes}
\usepackage[colorlinks=true, allcolors=blue]{hyperref}

\title{Infamous Equation 4}
\author{Jean P. Gibert}

\begin{document}
\maketitle

\section{Phenotypic evolution in two sites}
In this paper, we aim at understanding how straying may affect phenotypic evolution, and how in turn this evolutionary change may lead to changes in ecological dynamics as they unfold. To do so, we build upon ideas from classic papers (Lande 1976, most notably). In what follows, we show how the application of Lande's equation for a situation like the one we are keeping track of leads to overly-complicated equations of trait evolution that make our model difficult to track, and our understanding of the processes in play, difficult to achieve. 

\subsection{Lande's classic work}
To keep track of the evolution of a phenotypic trait $z$ with probability density function $p$ and mean $\overline{z}$ and variance $\sigma^{2}$, we need a few basic ingredients. First, we define the mean fitness,$\overline{W}$ of the population of individuals bearing the trait $z$ as:

\begin{equation}\label{1}
\overline{W}=\int_{} p(z)W(z)\,dz.
\end{equation}
We also define the mean phenotype in the population as  

\begin{equation}\label{2}
\overline{z}=\int_{} zp(z)\,dz,
\end{equation}
and the mean phenotype after selection (but before reproduction), $\overline{z}_{w}$ (Falconer 1960), as:

\begin{equation}\label{3}
\overline{z}_{w}=\frac{1}{\overline{W}}\int_{} zp(z)W(z)\,dz.
\end{equation}
Then, we can assess how the change in the phenotype $z$ occurs over time following
\begin{equation}\label{4}
\Delta z=z(t+1)-z(t),
\end{equation}
which can be rewritten as (Falconer 1960, Lande 1976):
\begin{equation}\label{5}
\Delta z=h^{2}(\overline{z}_{w}-z(t)),
\end{equation}
where $h^{2}$ is the heritability of the trait, and the difference $\overline{z}_{w}-z(t)$ is the classical "selection differential". In other words, \ref{5} is none other that the Breeder's equation. 

To understand how $z$ changes over time, we first need to understand how the mean fitness of the population changes with a change in the mean trait value or $\frac{\delta \overline{W}}{\delta \overline{z}}$ (i.e. the shape of the "adaptive landscape"). Using \ref{1} we have:

\begin{equation}\label{6}
\frac{\delta \overline{W}}{\delta \overline{z}}=\frac{\delta}{\delta \overline{z}}\int_{} p(z)W(z)\,dz.
\end{equation}
Using Leibniz rule we can pass the derivative under the integral sign, 
\begin{equation}\label{7}
\frac{\delta \overline{W}}{\delta \overline{z}}=\int_{} \frac{\delta p(z)}{\delta \overline{z}}W(z)\,dz,
\end{equation}
and assuming $p$ is Gaussian (i.e. of the form $\frac{1}{\sqrt{2 \pi \sigma^{2}}} e^{-\frac{1}{2}\frac{(z-\overline{z})^{2}}{\sigma^{2}}} $), we calculate the derivative of $p$ with respect to $\overline{z}$, which leads to:

\begin{equation}\label{8}
\frac{\delta \overline{W}}{\delta \overline{z}}=\int_{} \Big[\frac{z-\overline{z}}{\sigma^{2}}\Big] p(z)W(z)\,dz.
\end{equation}
We can now expand in \ref{8} to get,
\begin{equation}\label{9}
\frac{\delta \overline{W}}{\delta \overline{z}}=
\frac{1}{\sigma^{2}}\int_{}zp(z)W(z)\,dz-\frac{\overline{z}}{\sigma^{2}}\int_{}p(z)W(z)\,dz,
\end{equation}
which making use of equations \ref{1} and \ref{3}, becomes $\frac{\delta \overline{W}}{\delta \overline{z}}=
\frac{\overline{W}}{\sigma^{2}}\overline{z}_{w}-\frac{\overline{z}}{\sigma^{2}}\overline{W}$, which can be factored as $\frac{\delta \overline{W}}{\delta \overline{z}}=
\frac{\overline{W}}{\sigma^{2}}(\overline{z}_{w}-\overline{z})$. Using \ref{5}, we can rearrange this expression to obtain an equation that related the change in the average phenotype from one time step to the next to the amount of heritable variation, $\sigma^{2}h^{2}$, and the adaptive landscape $\frac{\delta \overline{W}}{\delta \overline{z}}$:
\begin{equation}\label{10}
\Delta \overline{z}=\frac{\sigma^{2}h^{2}}{\overline{W}}\frac{\delta \overline{W}}{\delta \overline{z}}=\sigma^{2}h^{2}\frac{\delta \ln\overline{W}}{\delta \overline{z}}.
\end{equation}
Equation \ref{10} (Lande 1976's equation 7), is a staple of evolutionary biology. It is also possible to use \ref{10} and \ref{4} to write a recurrence relationship that allows to simulating evolutionary change over time in a trait $z$ as:
\begin{equation}\label{11}
\overline{z}(t+1)=\overline{z}(t)+\sigma^{2}h^{2}\frac{\delta \ln\overline{W}}{\delta \overline{z}}.
\end{equation}

\subsection{Using Lande's approach for two populations}
In what follows we show how Lande's approach can be used in the case this paper focuses on (two populations with one trait but two different means in each site) to calculate the magnitude of the phenotypic change analytically, why it is not necessarily useful to use the obtained expression in a simulation context, and what we can do to circumvent that issue.

In our case, we have two populations that exchange migrants, with a normally distributed trait $z$ that controls recruitment (as explained in the main text), with each population having a different mean, $\mu_{1}$ and $\mu_{2}$, but same variance $\sigma^{2}$. For simplicity, we will focus on population 1, but all the results we present in what follows are analogous in population 2 as well.

When individuals from population 2 stray into population 1, and individuals of population 1 stray into population 2, the phenotypic distribution of trait $z$ in population 1 is a mixture distribution, $p(z)$, of the form:

\begin{equation}\label{12}
p(z)=\omega p(z,\mu_{1})+(1-\omega)p(z,\mu_{2}).
\end{equation}
Such distribution has a mean $\overline{z}=\omega \mu_{1}+(1-\omega)\mu_{2}$. As before, we define the mean fitness as in equation \ref{1}, only that now $p$ is a mixture distribution. Let us use Lande's approach to obtain an expression for the change in the mean phenotype of the mixture distribution from one time step to the next. We thus ask, what is the change in $\overline{W}$ with a change in $\overline{z}$? We can use Lande's approach all the way to equation \ref{7}, then replace $p(z)$ by the mixture (equation \ref{12}): 

\begin{equation}\label{13}
\frac{\delta \overline{W}}{\delta \overline{z}}=\int_{} \frac{\delta (\omega p(z,\mu_{1})+(1-\omega)p(z,\mu_{2}))}{\delta \overline{z}}W(z)\,dz.
\end{equation}
To take the derivative in equation \ref{13}, we need a change of variables, which is provided by the fact that $\overline{z}=\omega \mu_{1}+(1-\omega)\mu_{2}$:
\begin{equation} \label{14}
	\begin{cases} 
		\mu_{1}=\frac{\overline{z}-(1-\omega)\mu_{2}}{\omega}\\
		\mu_{2}=\frac{\overline{z}-\omega\mu_{1}}{1-\omega}\\
	\end{cases}.
\end{equation}
We can thus replace equation \ref{14} into \ref{13},
\begin{equation}\label{15}
\frac{\delta \overline{W}}{\delta \overline{z}}=\int_{} \frac{\delta (\omega p(z,\frac{\overline{z}-(1-\omega)\mu_{2}}{\omega})+(1-\omega)p(z,\frac{\overline{z}-\omega\mu_{1}}{1-\omega}))}{\delta \overline{z}}W(z)\,dz,
\end{equation} 
which, after expanding and taking the derivatives with respect to $\overline{z}$ becomes:

\begin{multline}\label{16}
\frac{\delta \overline{W}}{\delta \overline{z}}=\int_{} \omega\Bigg[\frac{z-\frac{\overline{z}-(1-\omega)\mu_{2}}{\omega}}{\omega \sigma^{2}}\Bigg] p(z,\frac{\overline{z}-(1-\omega)\mu_{2}}{\omega})W(z)\,dz + \\ + \int_{} (1-\omega)\Bigg[\frac{z-\frac{\overline{z}-\omega\mu_{1}}{1-\omega}}{(1-\omega) \sigma^{2}}\Bigg] p(z,\frac{\overline{z}-\omega\mu_{1}}{1-\omega})W(z)\,dz.
\end{multline}
To simplify notation, at this point we can come back to the original variables $\mu_{1}$ and $\mu_{2}$. We then expand and collect terms to get:
\begin{multline}\label{17}
\frac{\delta \overline{W}}{\delta \overline{z}}=\frac{1}{\sigma^{2}}\Bigg[\int_{}zp(z,\mu_{1})W(z)\,dz + \mu_{1}\int_{}p(z,\mu_{1})W(z)\,dz + \\ + \int_{}zp(z,\mu_{2})W(z)\,dz + \mu_{2}\int_{}p(z,\mu_{2})W(z)\,dz\Bigg].
\end{multline}
Using equations \ref{1} and \ref{3}, we can define the following quantities,

\begin{equation}\label{18}
\overline{W}_{1}=\int_{} p(z,\mu_{1})W(z)\,dz 
\end{equation}
\begin{equation}\label{19}
\overline{W}_{2}=\int_{} p(z,\mu_{2})W(z)\,dz  
\end{equation}
\begin{equation}\label{20}
\overline{z}_{w,1}=\frac{1}{\overline{W}}\int_{}zp(z,\mu_{1})W(z)\,dz
\end{equation}
\begin{equation}\label{21}
\overline{z}_{w,2}=\frac{1}{\overline{W}}\int_{}zp(z,\mu_{2})W(z)\,dz, 
\end{equation}
then replace them into equation \ref{17}:
\begin{equation}\label{22}
\begin{aligned}
\frac{\delta \overline{W}}{\delta \overline{z}} &=\frac{1}{\sigma^{2}}\Big[\overline{z}_{w,1} \overline{W}_{1} + \mu_{1}\overline{W}_{1} + \overline{z}_{w,2} \overline{W}_{2} + \mu_{2}\overline{W}_{2}\Big]  \\ 
&= \frac{1}{\sigma^{2}}\Big[ \overline{W}_{1}(\overline{z}_{w,1}-\mu_{1}) + \overline{W}_{2}(\overline{z}_{w,2}-\mu_{2})\Big].
\end{aligned}
\end{equation}
Using equation \ref{5} and rearranging terms we get:
\begin{equation}\label{23}
\overline{W}_{1}\Delta \mu_{1} + \overline{W}_{2}\Delta\mu_{2} = \sigma^{2}h^{2}\frac{\delta \overline{W}}{\delta \overline{z}} ,
\end{equation}
with $\overline{W}=\omega\overline{W}_{1}+(1-\omega)\overline{W}_{2}$. Equation \ref{23} relates how changes in either $\mu_{1}$ or $\mu_{2}$ lead to changes in fitness, and viceversa. However, it doesn't tell us anything as to how straying may lead to evolutionary change, and how that may feedback onto ecological dynamics. Because of this, we will once again use the change of variables in \ref{14}, to rewrite \ref{23} as:
\begin{equation}\label{24}
\overline{W}_{1}\Big(\frac{\Delta\overline{z}-(1-\omega)\Delta\mu_{2}}{\omega} \Big)+\overline{W}_{2}\Big(\frac{\Delta\overline{z}-\Delta\mu_{1}}{1-\omega} \Big)= \sigma^{2}h^{2}\frac{\delta \overline{W}}{\delta \overline{z}}, 
\end{equation}
which, can be simplified and rearranged to obtain an expression for the change in the mean phenotype of the mixture distribution,
\begin{equation}\label{25}
\Delta\overline{z}=\sigma^{2}h^{2}\frac{\delta \overline{W}}{\delta \overline{z}}\Big[\frac{(1-\omega)}{\omega}\overline{W}_{1}\Delta\mu_{2}+\frac{\omega}{1-\omega}\overline{W}_{2}\Delta\mu_{1} \Big]\frac{\omega}{\overline{W}_{1}+\overline{W}_{2}}.
\end{equation}
By multiplying \ref{25} by $\frac{\overline{W}}{\overline{W}}$, we can further rewrite the equation as:
\begin{equation}\label{26}
\Delta\overline{z}=\sigma^{2}h^{2}\frac{\delta \ln \overline{W}}{\delta \overline{z}}\Big[\frac{(1-\omega)}{\omega}\overline{W}_{1}\Delta\mu_{2}+\frac{\omega}{1-\omega}\overline{W}_{2}\Delta\mu_{1} \Big]\frac{\omega \overline{W}}{\overline{W}_{1}+\overline{W}_{2}},
\end{equation}
which has a similar form to Lande's equation (\ref{10}), but shows the explicit dependence of the change in the mean of the mix on the proportion of straying individuals. A similar expression can be derived for the other site. 

While some understanding can be gained using equation \ref{26}, and we now present that in our main text, the expression does not easily allow us to simulate the change in $\overline{z}$ over time because the change in $\mu_{1}$ and $\mu_{2}$ is difficult to track in closed form. Moreover, this equation only holds for the first time step, but as new migrant arrive and the trait distributions get updated by the influx of individuals from the other site, the equation does not hold anymore. 

This leaves us between a rock and a hard place: we can either track this change exactly for one generation, or we can do so through an approximation. Given that our paper aims to gain understanding about the feedback between evolutionary and ecological processes in salmon populations with spatial structure, we decided to simplify our problem by assuming that, as new migrants arrive, the mixture distribution can be locally approximated by a gaussian distribution whose mean equals the mean of the mix. Below, we show why this approximation, albeit coarse at first view, actually does a good job at describing the phenotypic distributions found in our model system under some conditions. Also, because of this assumption, we can use Lande's equation (here equation \ref{10}), without violating the original assumptions that led to its current form.




% For simplicity, let us assume the existence of a population of organisms with a normally distributed trait $z$, with mean $\mu$ and variance $\sigma^{2}$. Further assuming that the trait $z$ controls recruitment, $R$, and is under stabilizing selection, we show in the main text that recruitment is a function of the trait mean, $\mu$, of the form:

% \begin{equation}\label{recruitment_func}
% 	R(\mu)=\frac{r_{max}\tau}{\sqrt{\sigma^{2}+\tau^{2}}} e^{-\frac{(\theta-\mu)^{2}}{2(\sigma^{2}+\tau^{2})}}    .
% \end{equation}

% Under these assumptions, the evolution of $\mu$ over time is controlled by the following recurrence relationship:

% \begin{equation}\label{recurrenceONE}
% \mu_{t+1}=\mu_{t}+S,
% \end{equation}
% where $S$ has been shown by Lande (1976) to be equal to,

% \begin{equation}\label{selectionLande}
% \sigma^{2}h^{2}\frac{\partial \ln(W)}{\partial \mu_{t}} 
% \end{equation}
% with $\sigma^{2}h^{2}$ being the total heritable variation of trait $z$, and $W$ being the average fitness of the organisms. In this paper, we define our recruitment function as our measure of fitness, which allows us to write \ref{recurrenceONE} as:

% \begin{equation}\label{recurrenceTWO}
% \mu_{t+1}=\mu_{t}+\sigma^{2}h^{2}\frac{\partial \ln(R(\mu_{t}))}{\partial \mu_{t}} .
% \end{equation}

% Let us now assume the existence of two populations, as is done in the main text, where both population share the trait $z$, but with different means $\mu_1$ and $\mu_2$ (variance is kept the same for both populations). While we do not contend that equation \ref{recurrenceTWO} holds in a situation where $z$ is not normally distributed –it simply does not– we notice our simplifying assumption, once one population receives migrants from the second population (quoting from the main text): "we assume that the distribution resulting from the mix of remote and local individuals, following reproduction, is also normal with a mean value being that of the mixed-normal". Under these conditions, recruitment in the focal population (population 1) depends on: 1) recruitment from the local individuals, which now make up a proportion $\omega$ of the population, and 2) recruitment from the individuals that came from the other populations, which now make up a proportion $1-\omega$ of the focal population. In other words, recruitment in the focal population is now $\omega R(\mu_{1})+(1-\omega) R(\mu_{1})$. Because we assume that immediately after reproduction the ensuing recruits have a trait distribution that is normally distributed, that allows us to simply replace $\omega R(\mu_{1})+(1-\omega) R(\mu_{1})$ as our fitness in \ref{recurrenceTWO}, to obtain equation XXX
%  of the main text. 





















\end{document}